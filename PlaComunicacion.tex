\documentclass{article}
\oddsidemargin=1cm
\usepackage[spanish]{babel}%paquete principal2
\usepackage[utf8]{inputenc}%paquete principal2
\usepackage{latexsym} % Simbolos 
\usepackage{graphicx} % Inclusion de graficos. Soporte para figura 
\usepackage{hyperref} % Soporte hipertexto
\usepackage{subfigure}
\usepackage{epsfig}
\usepackage{graphicx}
\usepackage{rotating}
\usepackage{epstopdf}
\usepackage{ctable}
\usepackage{longtable}
\usepackage{color}
\usepackage{colortbl}
\usepackage{tabularx,colortbl}
\usepackage{setspace}
\usepackage{threeparttable}
\usepackage{multirow}
\usepackage{pdflscape}
\usepackage{anysize}
%\usepackage[authoryear]{natbib}
\usepackage{breakurl}
\usepackage{url}
\usepackage{amssymb}
\usepackage{graphicx}
\usepackage[centertags]{amsmath}
\usepackage{amsthm}
\usepackage{array}
\usepackage{times}
\usepackage[left=1in, right=1in, top=1in, bottom=1in]{geometry}
\usepackage{rotating}
\usepackage{amstext}
\usepackage{pdfpages}
\usepackage{amsbsy}
\usepackage{amsopn}
\usepackage{eucal}
\usepackage{sectsty}
\usepackage{titlesec}
\usepackage[capposition=top]{floatrow}
\usepackage{pdfpages}
\usepackage{apacite}
%\usepackage[singlespacing]{setspace}
\usepackage[labelsep=period,font={bf},textfont={normalsize},labelfont={normalsize}]{caption}
%\usepackage[longnamesfirst,comma]{natbib}
%\usepackage{iadb}
\title{\textbf{Estrategia de comunicación} \\ ''Competencia Boliviana de Posters Estadísticos ''}
\author{Fundación ARU}
\date{2018-2019 }
\begin{document} % Inicio del documento
\maketitle

\hrule
\hrule
\newpage

%%%%%%%%%%%%%%%%%%%%%%%%%%%%%%%%%%%%%%%%
\section{Introducción}

El presente plan de comunicación propone, aportar al logro de los objetivos, en la competencia del proyecto ''Alfabetización Estadística''. Plan dirigido a captar la atención de los estudiantes de colegios y universidades, para que confirmen su participación como grupos en la Competencia Boliviana de Posters Estadísticos 2018-2019.\\

En el se podrá encontrar una descripción de la organización, y desarrollo de la competencia llevada por primera vez en Bolivia.


\section{Objetivos}


\begin{itemize}
\item Promocionar la competencia para lograr despertar el interés en estudiantes y docentes por la misma. 

\item Difundir mediante diversos canales de comunicación los requerimientos técnicos y de participación en cada etapa de la competencia.
\end{itemize} 

\section{Definición de actores objetivo}


La competencia esta dirigida a estudiantes de colegio y estudiantes del pregrado en las universidades públicas y privadas de toda Bolivia, asimismo los estudiantes deben contar con el apoyo de un docente que acompañe su trabajo y les sirva de guía, en este sentido se prevee que los actores clave sean los estudiantes antes mencionados, docentes y centros de interés que al recibir toda ésta información puedan formar parte del grupo patrocinador en futuras actividades.

\section{Implementación}

Dado los objetivos esperados, se propone la siguiente linea de implementación para el proyecto:

\begin{enumerate}
\item Presentación del proyecto
\begin{itemize}
\item Audiencia esperada:
\begin{itemize}
\item Representante de la Iniciativa
\item Institución Organizadora
\item Patrocinadores
\item Estudiantes de colegios
\item Estudiantes universitarios
\end{itemize}
\item Fecha tentativa:
\begin{itemize}
\item 22 de Junio
\end{itemize}
\item Objetivo
\begin{itemize}
\item Presentar el proyecto
\item Presentar la convocatoria
\item Presentar a las instituciones involucradas
\item Presentar los medios disponibles para la cobertura en participación
\begin{itemize}
\item Página de la iniciativa 
\item Canal de facebook
\item Canal de twitter
\end{itemize}
\end{itemize}
\item Herramienta de apoyo
\begin{itemize}
\item Armar un stand de información
\end{itemize}
\end{itemize}
\item Difusión en sitio web redes sociales, dirigido al público en general 
\begin{itemize}
\item Audiencia esperada:
\begin{itemize}
\item Estudiantes de colegios
\item Estudiantes universitarios
\end{itemize}
\item Fecha tentativa:
\begin{itemize}
\item 22 de Junio del 2018 - 14 de Enero del 2019
\end{itemize}
\item Objetivo
\begin{itemize}
\item Mantener informado al público del proyecto
\item Presentar fuentes de información disponibles
\item Proponer tutoriales de manejo de información
\begin{itemize}
\item Página de la iniciativa 
\item Canal de facebook
\item Canal de twitter
\item Programación de talleres de apoyo a participantes
\end{itemize}
\end{itemize}
\item Herramienta de apoyo
\begin{itemize}
\item Internet
\end{itemize}
\end{itemize}

\item Spot publicitario 
\begin{itemize}
\item Audiencia esperada:
\begin{itemize}
\item Estudiantes de colegios
\item Estudiantes universitarios
\item Profesores y docentes
\end{itemize}
\item Fecha tentativa:
\begin{itemize}
\item 22 de Junio del 2018 - Agosto del 2018
\end{itemize}
\item Objetivo
\begin{itemize}
\item Crear un corto de video que transmita la convocatoria y publicarlo en:
\begin{itemize}
\item Página de la iniciativa 
\item Canal de facebook
\item Canal de twitter
\end{itemize}
\end{itemize}
\item Herramienta de apoyo
\begin{itemize}
\item Internet
\end{itemize}
\end{itemize}

\item Generación de materiales
\begin{itemize}
\item Audiencia esperada:
\begin{itemize}
\item Estudiantes de colegios
\item Estudiantes universitarios
\end{itemize}
\item Fecha tentativa:
\begin{itemize}
\item 22 de Junio del 2018 - Agosto del 2018
\end{itemize}
\item Objetivo
\begin{itemize}
\item Motivar la participación de los estudiantes
\item Informar a la audiencia de la convocatoria
\item Publicar posters informativos
\item Publicar folletos de información y convocatoria
\end{itemize}
\item Herramienta de apoyo
\begin{itemize}
\item Personal de la entidad organizadora y las instituciones patrocinadoras
\end{itemize}
\end{itemize}
\item Cierre de clausura
\begin{itemize}
\item Audiencia esperada:
\begin{itemize}
\item Representante de la Iniciativa
\item Institución Organizadora
\item Patrocinadores
\item Estudiantes de colegios
\item Estudiantes universitarios
\end{itemize}
\item Fecha tentativa:
\begin{itemize}
\item 29 de Marzo del 2019
\end{itemize}
\item Objetivo
\begin{itemize}
\item Presentar a las instituciones involucradas
\item Dar ha conocer el alcance de la iniciativa
\item Presentar a los ganadores de la convocatoria
\begin{itemize}
\item Página de la iniciativa 
\item Canal de facebook
\item Canal de twitter
\end{itemize}
\item Premiar a los ganadores de la competencia
\end{itemize}
\item Herramienta de apoyo
\begin{itemize}
\item Internet, correo electrónico
\end{itemize}
\end{itemize}
\end{enumerate}


\section{Recursos económicos}

\section{Recursos humanos}

El equipo de trabajo necesario para el buen desarrollo delas actividades involucradas con la Competencia Boliviana de Posters estadísticos es el siguiente: \\

\begin{itemize}
\item Representante de la iniciativa: Alvaro Chirino.  
\item Instituciones Organizadoras: Fundación ARU, Instituto Nacional de Estadística.
\item Entidades Patrocinadoras: INE, Fundacion ARU
\item Comité evaluador: INE, Fundación ARU, Carrera de Estadística UMSA
\item Responsable de medios de difusión en sitio web:  Fundación ARU, INE, patrocinadores. 
\item Responsable del manejo de redes sociales: Fundación ARU.
\item Responsable de publicación de carteles y afiches de Información: INE.
\item Recepción de poster físicos: INE.
\item Recepción de poster digitales: Fundación ARU.
\item Organizador del lanzamiento de la competencia: Fundación ARU.
\item Organizador del Evento de premiación: Fundación ARU.

\end{itemize}


\section{Cronograma de actividades}


\begin{table}[h]
\begin{center}
\scalebox{0.9}{
\begin{tabular}{|p{10cm}|p{4cm}|}
\hline
{\bf  Actividad.} & {\bf Tiempos}  \rule[-0.2cm]{0cm}{0.6cm} \\
\hline
1. Diseño de materiales de difusión(carteles, afiches, páginas web, etc.)  & Marzo, abril y mayo de 2018   \rule[-0.2cm]{0cm}{0.6cm}\\
\hline
2. Lanzamiento de la Competencia en Acto público  &  22 de junio del 2018   \rule[-0.2cm]{0cm}{0.6cm}\\
\hline 
3. Publicación de Carteles y Afiches  & Junio, julio y agosto del 2018     \rule[-0.2cm]{0cm}{0.6cm}\\
\hline 
4. Inscripción de  equipos participantes   & 22 de junio 2018 al 14 de enero del 2019  \rule[-0.2cm]{0cm}{0.6cm}\\
\hline
5. Periodo de envió y registro del póster &  22 de junio de 2018 al 14 de enero del 2019  \rule[-0.2cm]{0cm}{0.6cm}\\
\hline
6. Organización de talleres de orientación &  Bajo coordinación con los equipos participantes \rule[-0.2cm]{0cm}{0.6cm}\\
\hline
7. Evaluación de los póster &  15 de enero de 2018 a 20 de marzo del 2019    \rule[-0.2cm]{0cm}{0.6cm}\\
\hline
8. Evento de premiación a los grupos de los posters ganadores  &  29 de marzo del 2019    \rule[-0.2cm]{0cm}{0.6cm}\\
\hline 
\end{tabular}}
\end{center}
\end{table}



\end{document}