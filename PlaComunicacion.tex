
\documentclass{article}

\oddsidemargin=1cm
\usepackage[spanish]{babel}%paquete principal2
\usepackage[utf8]{inputenc}%paquete principal2
\usepackage{latexsym} % Simbolos 
\usepackage{graphicx} % Inclusion de graficos. Soporte para figura 
\usepackage{hyperref} % Soporte hipertexto
\usepackage{subfigure}
\usepackage{epsfig}
\usepackage{graphicx}
\usepackage{rotating}
\usepackage{epstopdf}
\usepackage{ctable}
\usepackage{longtable}
\usepackage{color}
\usepackage{colortbl}
\usepackage{tabularx,colortbl}
\usepackage{setspace}
\usepackage{threeparttable}
\usepackage{multirow}
\usepackage{pdflscape}
\usepackage{anysize}
%\usepackage[authoryear]{natbib}
\usepackage{breakurl}
\usepackage{url}
\usepackage{amssymb}
\usepackage{graphicx}
\usepackage[centertags]{amsmath}
\usepackage{amsthm}
\usepackage{array}
\usepackage{times}
\usepackage[left=1in, right=1in, top=1in, bottom=1in]{geometry}
\usepackage{rotating}
\usepackage{amstext}
\usepackage{pdfpages}
\usepackage{amsbsy}
\usepackage{amsopn}
\usepackage{eucal}
\usepackage{sectsty}
\usepackage{titlesec}
\usepackage[capposition=top]{floatrow}
\usepackage{pdfpages}
\usepackage{apacite}
%\usepackage[singlespacing]{setspace}
\usepackage[labelsep=period,font={bf},textfont={normalsize},labelfont={normalsize}]{caption}
%\usepackage[longnamesfirst,comma]{natbib}
%\usepackage{iadb}
\title{\textbf{Plan de Comunicación} \\ ''Unidad de Datos''}
\author{Fundación ARU}
\date{Mayo 2018}
\begin{document} % Inicio del documento
\maketitle

\hrule
\hrule
\newpage

%%%%%%%%%%%%%%%%%%%%%%%%%%%%%%%%%%%%%%%%
\section{Estrategia}

\begin{itemize}
\item Introducción: resumen del proyecto; enfatizar el valor agregado
\item Misión y visión
\item Uno o dos objetivos claros u objetivos
\item Audiencias clave, mensajes, canales
\item Calendario de actividades clave con fechas
\item Recursos: quién hará el trabajo y quién pagará
\item Riesgos y mitigación
\item Medios de evaluación
\item Aprobaciones / proceso de aprobación
\end{itemize}

Y finalmente….
\begin{itemize}
\item Guardar y administrar versiones de su estrategia
\item Copias electrónicas e impresas del material producido
\item Registro de evaluación cuantitativa y cualitativa
\item Mantener una lista de contactos
\item Compartir las mejores prácticas con los compañeros
\item Publicite su éxito
\item ¡Es un buen RP para nuestra profesión!
\end{itemize}

Recursos internos
\begin{itemize}
\item El sitio web de la Dirección de Asuntos Públicos ha mucha información:
www.ox.ac.uk/public-affairs
\item Considerar a los oficiales de comunicaciones Red, red de medios sociales, divisional redes y recursos, y el Colegio Red de comunicaciones
\item Consulte la Guía de estilo, Guía de estilo digital y pautas de marca para asegurarse su ejecución de comunicaciones se encuentra con la marca estándares y mejores prácticas recomendaciones
\end{itemize}

Estructura de Plan de Comunicación
\begin{itemize}
\item Definición del Alcance
\item Objetivos
\item Público(s) objetivo
\item Mensajes
\item Acciones
\item Mecanismos de retroalimentación
\item Cronograma o calendario
\item Presupuesto
\end{itemize}

\section{Introducción}

Fundación ARU trabaja, en función de dos áreas: Investigación y Recolección de datos. La Unidad de Datos (UD) es el grupo encargado de la recolección de datos. La UD trabaja desarrollando en cada proyecto la: temática, muestreo, sistema y operativo de campo. Los proyectos desarrollados, son adquiridos de la participación a convocatorias de entidades públicas o privadas y por invitaciones directas a la fundación.\\

La UD siempre ha buscado innovar y mejorar los instrumentos y materiales a través de la experiencia adquirida con los proyectos y el avance de la tecnología. La actualización constante de los protocolos de trabajo hacen notar la inquietud por buscar la optimización de los recursos disponibles.\\                           

El Proyecto Internacional de la Alfabetización Estadística (ISLP) está bajo el patrocinio de la Asociación Internacional para la Educación Estadística (IASE), la sección de educación del Instituto Internacional de Estadística (ISI).\\

La primera Competencia Internacional de Alfabetización Estadística del ISLP se lanzó en el año 2007. A la fecha, alrededor de 120 Países se encuentran representados dentro del ISLP por sus coordinadores nacionales\\

A finales de 2017 Alvaro Chirino fue incorporado como el coordinador nacional en Bolivia, (Noveno país en sudamerica).

A inicios de 2018 el ISLP lanzo la Competencia Internacional de
Posters, donde se invita a estudiantes de escuelas y universidades de todo el mundo a diseñar un póster estadístico.

%%%%%%%%%%%%%%%%%%%%%%%%%%%%%%%%%%%%%%%%
\section{Objetivo}
El proyecto de Alfabetizacion estadistica atraves del concurso de Posters persigue los siguientes objetivos 

Promover acciones educativas que favorezcan el desarrollo de la
alfabetización estadística.
Desarrollar habilidades de indagación y divulgación cientíca a través de un póster con herramientas estadísticas.
Fomentar el intercambio de experiencias entre los diferentes actores que intervengan en un proceso de indagación cientíca.
Favorecer la consolidación de equipos de aprendizaje, que contribuyan a estrechar lazos entre la indagación y el contexto.




La UD tiene por objetivo diseñar y generar instrumentos y materiales que aporten mejoras en la recolección de información, para asegurar que los productos resultantes (base de datos, conclusiones) contengan información veráz, de calidad y oportuna, para de esta manera pretender realizar algún aporte al entorno social.
\section{La Unidad de Datos (UD)}


La unidad de datos desarrolla su trabajo a través de la participación de los responsables en cada una de las áreas designadas.\\

La UD está compuesta por: el coordinador, el responsable de temática, el responsable de muestreo, el responsable de sistemas, el responsable de operativo de campo y el responsable de la elaboración de los metadatos. Cada responsable se encarga de elaborar, diseñar, gestionar los productos necesarios para la efectiva operación del proyecto,  así como también dar a conocer los detalles sobre ellos a todo el equipo, el reporte final de las actividades y productos  marcan el fin de su intervensión. Algunos productos generados  sirven de insumo para generar otros productos.\\

Existen dos campos de trabajo de la unidad de datos:

\begin{itemize}
\item Recolección de información 
\item Actualización de instrumentos para la captura y transmisión de información
\end{itemize}

La recolección de información comienza con el desenvolvimiento de la pregunta de investigación pasando por todas las áreas de los responsables de la UD y el desarrollo de instrumentos para la captura y transmisión de información empieza con la terminación de los proyectos de investigación.

\subsection{Recolección de información}

Al interior de la Fundación ARU, durante la ejecución de proyectos de recolección de información la Unidad de Datos interactúa con la Unidad de Investigación y la Unidad Administrativa y Gestión de Comunicaciones.

%\begin{center}
%\includegraphics[scale=0.7]{EFuncionamiento.png}
%\end{center}

\subsubsection{Unidad de Investigación}

Esta unidad trabaja conjuntamente con el responsable de temática de la UD, en las siguientes actividades; 

\begin{enumerate}
\item Compilación de material bibliográfico
\item Diseño del cuestionario así como su aprobación con la entidad contratante
\item Glosario y elaboración de fichas técnicas
\item Depuración de la base con el cálculo de indicadores
\item Elaboración del documento final, donde se describirá los resultados y conclusiones a través de tablas, gráficos e informes.
\end{enumerate}

\subsubsection{Unidad Administrativa y Gestión de Comunicaciones}

Esta unidad trabaja en coordinación con el responsable delegado para ser el enlace con el equipo de trabajo de la UD. La coordinación es necesaria entre ambas partes, dado que el receptor del contrato así como los desembolsos programados son manejados por el oficial administrativo y las actividades ha desarrollar en la recolección de información son programadas por la UD, así como también el presupuesto a ejecutarse.\\

Las actividades a coordinar entre ambas partes son:

\begin{enumerate}
\item Ajuste de presupuesto 
\item Compra de materiales para operativo de campo (material de escritorio, material logístico)
\item Desembolso de recursos económicos para operativo de campo (viático, transporte)
\item Desembolso de recursos económicos para planilla de pagos (personal)
\item Emisión de contratos para el personal (encuestadores y supervisores) 
\item Envió y recepción de materiales (cuestionarios, mapas, saldos de material de escritorio, saldo de incentivos)
\item Emisión de certificados de trabajo
\item Rendición y aprobación de los gastos de operativo - Unidad de Administración y Finanzas - Unidad de Datos
\end{enumerate}

\subsubsection{Funciones de los responsables de la UD}
%\begin{center}
%\includegraphics[scale=0.7]{trab3.png}
%\end{center}

Las funciones de los responsables de la UD están muy relacionados con las actividades y productos que deben generar para la operativización del proyecto a desarrollarse. 

\subsubsection{Deberes de los responsables de la UD}

Los deberes de los responsables estan relacionados con las actividades que deben realizar cuando empieza a ejecutarse un proyecto.

\begin{itemize}
\item Coordinador de la UD
\begin{enumerate}
\item Actuar como enlace entre la entidad contratante y la UD. 
\item Informar los detalles y características del proyecto a la UD.
\item Supervisar el avance de las actividades del proyecto
\end{enumerate}
\item Responsable de muestreo
\begin{enumerate}
\item Diseño muestral
\item Muestra
\item Factores de expansión de la muestra 
\item Instrumentos de selección de muestra en campo
\item Actualización de protocolo de trabajo
\end{enumerate}
\item Responsable de temática
\begin{enumerate}
\item Recopilación bibliográfica de la temática del proyecto
\item Matriz de variables, fichas técnicas
\item Cuestionario
\item Diagrama de flujo del cuestionario 
\item Tabla de control de contenido del cuestionario(control de calidad)
\item Material de capacitación para personal (diapositivas, conceptos)
\item Actualización de protocolo de trabajo
\end{enumerate}
\item Responsable de operativo de campo
\begin{enumerate}
\item Cronograma de actividades de la UD para con el proyecto
\item Ajuste de presupuesto para operativo de campo
\item Preparación de la logística para el operativo de campo
\item Planificación de la carga de trabajo (personal)
\item Material cartográfico (desplazamiento en campo)
\item Material de capacitación para personal (diapositivas, manual)
\item Actualización de protocolo de trabajo
\end{enumerate}
\item Responsable de sistemas
\begin{enumerate}
\item Plantilla de entrada de datos (tablet)
\item Diseño de sistema de información
\item Diseño de plataforma de entrada de datos (envió en línea)
\item Material de capacitación para personal (diapositivas, protocolo)
\item Actualización de protocolo de trabajo
\end{enumerate}
\item Responsable de metadatos
\begin{enumerate}
\item Compilación de datos de los proyectos
\item Armado de repositorio de productos y materiales de los proyectos
\item Redacción del resumen de los proyectos
\item Actualización de información para la plataforma de  proyectos
\item Actualización de protocolo de trabajo
\end{enumerate}
\end{itemize}



\subsection{Actualización de instrumentos para la captura y transmisión de información}

Cada responsable de la UD al terminar su intervención en la recolección de información de un determinado proyecto, destina su tiempo a:

\begin{itemize}
\item Coordinador de UD
\begin{enumerate}
\item Organizar e impartir cursos o talleres de actualización
\item Creación de un directorio de convocatorias de posibles consultorias
\item Recopilar y centralizar la información para las propuestas
\item Redacción de documentos de investigación 
\end{enumerate}
\item Responsable de muestreo
\begin{enumerate}
\item Armonización de marcos muestrales
\item Apoyo a la unidad de investigación (manejo de base de datos)
\item Revisión de información de fuentes secundarias 
\end{enumerate} 
\item Responsable de temática
\begin{enumerate}
\item Apoyo a la unidad de investigación (manejo de base de datos)
\item Recopilación de información de fuentes secundarias 
\item Construcción de repositorio de temáticas sociales
\end{enumerate} 
\item Responsable de operativo de campo
\begin{enumerate}
\item Digitalización de mapas
\item Elaboración de mapas temáticos
\item Actualización de protocolos de campo 
\item Apoyo a la unidad de investigación (manejo de base de datos)
\item Recopilación de información de fuentes secundarias 
\item Desarrollo de plataforma de devolución de información
\end{enumerate} 
\item Responsable de sistemas
\begin{enumerate}
\item Mantenimiento y actualización al sistema de envió de información
\item Respaldo al sistema de cada proyecto, mantenimiento 
\item Mantenimiento de cuentas correo del sistema (tabletas y servidor)
\item Recopilación de información de fuentes secundarias 
\item Apoyo a la unidad de investigación (manejo de base de datos)
\item Desarrollo de plataforma de devolución de información
\end{enumerate} 
\item Responsable de metadatos
\begin{enumerate}
\item Compilación de datos, productos, instrumentos y materiales de los proyecto
\item Redacción del resumen de los proyectos
\end{enumerate} 
\end{itemize} 



\section{Instrumentos de la UD}

La UD desenvuelve su trabajo con el empleó de los siguientes instrumentos:

\begin{enumerate}
\item Servidor: Este es el receptor y contenedor de la información en línea; es el conducto de consultas para la devolución de información (REDATAM).  
\item Computador portatil: Instrumento usado para presentaciones (capacitación de personal), seguimiento de captura de información (cuestionario digital) y generación de documentos (certificados, cartas, etc.)    
\item Tabletas: Empleadas para la captura de información (cuestionarios digitales), evidencia fotográfica y comunicación con el personal de campo.
\item GPS: Instrumentos usados para el cálculo de posición de las unidades de observación entrevistadas.
\end{enumerate} 

\section{Consideraciones}

Los responsables de la UD deben tomar ciertas consideraciones con la labor que desarrollan:

\begin{enumerate}
\item Los productos de los responsables deben ser compartidos entre la totalidad de los miembros que componen el equipo de trabajo.
\item Los responsables deben participar de las actividades que asi lo requieran en las distintas áreas (prueba piloto, capacitación)
\item El presente documento es una aguía de la UD, que es complementada con los protocolos de cada área para un mejor conocimiento del funcionamiento de trabajo de cada responsable
\end{enumerate}

\section{Anexos}
Protocolo
\begin{itemize}
\item Temática - Disco ARU: BSMN-UnidadDatos-Tematica-Protocolo2016
\item Muestreo - Disco ARU: BSMN-UnidadDatos-Muestreo-Protocolo2016
\item Operativo de Campo - Disco ARU: BSMN-UnidadDatos-OperativoCampo-Protocolo2016
\item Sistemas - Disco ARU: BSMN-UnidadDatos-Sistemas-Protocolo2016
\item Metadatos - Disco ARU: BSMN-UnidadDatos-Metadatos-Protocolo2016
\end{itemize}

\end{document}