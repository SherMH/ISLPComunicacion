
\documentclass{article}

\oddsidemargin=1cm
\usepackage[spanish]{babel}%paquete principal2
\usepackage[utf8]{inputenc}%paquete principal2
\usepackage{latexsym} % Simbolos 
\usepackage{graphicx} % Inclusion de graficos. Soporte para figura 
\usepackage{hyperref} % Soporte hipertexto
\usepackage{subfigure}
\usepackage{epsfig}
\usepackage{graphicx}
\usepackage{rotating}
\usepackage{epstopdf}
\usepackage{ctable}
\usepackage{longtable}
\usepackage{color}
\usepackage{colortbl}
\usepackage{tabularx,colortbl}
\usepackage{setspace}
\usepackage{threeparttable}
\usepackage{multirow}
\usepackage{pdflscape}
\usepackage{anysize}
%\usepackage[authoryear]{natbib}
\usepackage{breakurl}
\usepackage{url}
\usepackage{amssymb}
\usepackage{graphicx}
\usepackage[centertags]{amsmath}
\usepackage{amsthm}
\usepackage{array}
\usepackage{times}
\usepackage[left=1in, right=1in, top=1in, bottom=1in]{geometry}
\usepackage{rotating}
\usepackage{amstext}
\usepackage{pdfpages}
\usepackage{amsbsy}
\usepackage{amsopn}
\usepackage{eucal}
\usepackage{sectsty}
\usepackage{titlesec}
\usepackage[capposition=top]{floatrow}
\usepackage{pdfpages}
\usepackage{apacite}
%\usepackage[singlespacing]{setspace}
\usepackage[labelsep=period,font={bf},textfont={normalsize},labelfont={normalsize}]{caption}
%\usepackage[longnamesfirst,comma]{natbib}
%\usepackage{iadb}
\title{\textbf{Estrategia de comunicación} \\ ''Competencia Boliviana de Posters Estadísticos ''}
\author{Fundación ARU}
\date{2018-2019 }
\begin{document} % Inicio del documento
\maketitle

\hrule
\hrule
\newpage

%%%%%%%%%%%%%%%%%%%%%%%%%%%%%%%%%%%%%%%%
\section{Introducción}

El presente plan de comunicación propone, aportar al logro de los objetivos, de la competencia del proyecto ''Alfabetización Estadística''. Está dirigido a captar la atención de los estudiantes de colegios y universidades, para que confirmen sus grupos de trabajo y participen de la Competencia Boliviana de Posters Estadísticos 2018-2019.\\

En el se podrá encontrar una descripción de la organización asi como los planes para el desarrollo de la competencia llevada por primera vez en Bolivia, a través por el representante de la iniciativa y la institución organizadora Fundación ARU.



\section{Objetivos}


\begin{itemize}
\item Promocionar la competencia para lograr despertar el interés en estudiantes y docentes por la misma. 

\item Difundir mediante diversos canales de comunicación los requerimientos técnicos y de participación en cada etapa de la competencia.
\end{itemize} 

\section{Definicion de actores objetivo}


Se ha definido que la competencia esté dirigida a estudiantes de colegio y estudiantes del pregrado en las universidades, asimismo deben contar con un docente que acompañe su trabajo, en este sentido se prevee que los actores clave sean los estudiante antes mencionados, docentes y centros de interés que al recibir toda ésta información puedan formar parte del grupo patrocinador en futuras ocaciones. \\

Propuesta



\section{Implementación}

Dado los objetivos esperados, se propone la siguiente linea de implementación para el proyecto:

\begin{enumerate}
\item Presentación del proyecto
\begin{itemize}
\item Audiencia esperada:
\begin{itemize}
\item Representante de la Iniciativa
\item Institución Organizadora
\item Patrocinadores
\item Estudiantes de colegios
\item Estudiantes universitarios
\end{itemize}
\item Fecha tentativa:
\begin{itemize}
\item 22 de Junio
\end{itemize}
\item Objetivo
\begin{itemize}
\item Presentar el proyecto
\item Presentar la convocatoria
\item Presentar a las instituciones involucradas
\item Presentar los medios disponibles para la cobertura en participación
\begin{itemize}
\item Página de la iniciativa 
\item Canal de facebook
\item Canal de twitter
\end{itemize}
\end{itemize}
\item Herramienta de apoyo
\begin{itemize}
\item Armar un stand de información
\end{itemize}
\end{itemize}

\item Difusión en sitio web redes sociales, dirigido al público en general 
\begin{itemize}
\item Audiencia esperada:
\begin{itemize}
\item Estudiantes de colegios
\item Estudiantes universitarios
\end{itemize}
\item Fecha tentativa:
\begin{itemize}
\item 22 de Junio del 2018 - 14 de Enero del 2019
\end{itemize}
\item Objetivo
\begin{itemize}
\item Mantener informado al público del proyecto
\item Presentar fuentes de información disponibles
\item Proponer tutoriales de manejo de información
\begin{itemize}
\item Página de la iniciativa 
\item Canal de facebook
\item Canal de twitter
\item Programación de talleres de apoyo a participantes
\end{itemize}
\end{itemize}
\item Herramienta de apoyo
\begin{itemize}
\item Internet
\end{itemize}
\end{itemize}

\item Diseño de materiales
\begin{itemize}
\item Audiencia esperada:
\begin{itemize}
\item Estudiantes de colegios
\item Estudiantes universitarios
\end{itemize}
\item Fecha tentativa:
\begin{itemize}
\item 22 de Junio del 2018 - 26 de Octubre del 2018
\end{itemize}
\item Objetivo
\begin{itemize}
\item Motivar la participación de los estudiantes
\item Informar a la audiencia de la convocatoria
\end{itemize}
\item Herramienta de apoyo
\begin{itemize}
\item Personal de la entidad organizadora y las instituciones patrocinadoras
\end{itemize}
\end{itemize}

\item Cierre de clausura

\begin{itemize}
\item Audiencia esperada:
\begin{itemize}
\item Representante de la Iniciativa
\item Institución Organizadora
\item Patrocinadores
\item Estudiantes de colegios
\item Estudiantes universitarios
\end{itemize}
\item Fecha tentativa:
\begin{itemize}
\item 29 de Marzo del 2019
\end{itemize}
\item Objetivo
\begin{itemize}
\item Presentar a las instituciones involucradas
\item Dar ha conocer el alcance de la iniciativa
\item Presentar a los ganadores de la convocatoria
\begin{itemize}
\item Página de la iniciativa 
\item Canal de facebook
\item Canal de twitter
\end{itemize}
\item Premiar a los ganadores de la competencia
\end{itemize}
\item Herramienta de apoyo
\begin{itemize}
\item Internet, correo electrónico
\end{itemize}
\end{itemize}


\end{enumerate}



\section{Recursos económicos}

\section{Recursos humanos}

\section{Cronograma de actividades}

%\section{Estrategia}

%\begin{itemize}
%\item Introducción: resumen del proyecto; enfatizar el valor agregado
%\item Misión y visión
%\item Uno o dos objetivos claros u objetivos
%\item Audiencias clave, mensajes, canales
%\item Calendario de actividades clave con fechas
%\item Recursos: quién hará el trabajo y quién pagará
%\item Riesgos y mitigación
%\item Medios de evaluación
%\item Aprobaciones / proceso de aprobación
%\end{itemize}

%Y finalmente….
%\begin{itemize}
%\item Guardar y administrar versiones de su estrategia
%\item Copias electrónicas e impresas del material producido
%\item Registro de evaluación cuantitativa y cualitativa
%\item Mantener una lista de contactos
%\item Compartir las mejores prácticas con los compañeros
%\item Publicite su éxito
%\item ¡Es un buen RP para nuestra profesión!
%\end{itemize}

%Recursos internos
%\begin{itemize}
%\item El sitio web de la Dirección de Asuntos Públicos ha mucha información:
%www.ox.ac.uk/public-affairs
%\item Considerar a los oficiales de comunicaciones Red, red de medios sociales, divisional redes y recursos, y el Colegio Red de comunicaciones
%\item Consulte la Guía de estilo, Guía de estilo digital y pautas de marca para asegurarse su ejecución de comunicaciones se encuentra con la marca estándares y mejores prácticas recomendaciones
%\end{itemize}

%Estructura de Plan de Comunicación
%\begin{itemize}
%\item Definición del Alcance
%\item Objetivos
%\item Público(s) objetivo
%\item Mensajes
%\item Acciones
%\item Mecanismos de retroalimentación
%\item Cronograma o calendario
%\item Presupuesto
%\end{itemize}


\end{document}